\documentclass[12pt,letterpaper]{article}
\author{Gerald Nilles}
\title{Sous Vide Cooker}

\usepackage{graphicx}


\begin{document}

%% Create title based on head \title info
\maketitle

%% Create TOC based on sections below
\tableofcontents

\newpage

%%%%%%%%%%%%%%%%%%%%%%%%

\section{Introduction}
The goal of this project is to create a cheap Sous Vide cooking system.
\subsection{What is Sous Vide}
Sous Vide is a method of cooking meat developed by the French.
With this method, the meat, seasonings and marinades are put in a sealed plastic bag (Vaccum bags are preferred by Zip-Lock bags are ok).
This bag is submerged in a pool of water that is precisly regulated to the meat's desired cooking temperature.
After several  hours, the meat's internal temperature will match that of the water and the meat will be perfectly cooked.\\

\subsection{Why Sous Vide}
By cooking the meat in a closed environment, all of the juices are sealed in and the meat will be extremely soft and juicy.
When finished, the meat is usually quickly seared with a hot pan or a creme brulee torch.
This will give the meat a carmelized taste and will make it look normal.


\section{Hardware Design}
Regulating the temperature requires a homemade circuit.
This circuit will use heating coils to warm the water, a thermocouple to monitor the water temperature, and a water pump to circuitlate the water.
All of these devices will be controlled by a microcontroller unit (MCU).
This MCU will turn on and off the heating coild and pump to maintain a set temperature.
The temperature will be displayed on a 2.5 digit LCD display.
\subsection{5V AC/DC Converter}
The wall voltage is alternating 120 Volts RMS (170 Volts Peak to Peak).  
This voltage will be good for powering the pump and the heating coil, but will be way to high for the MCU.
Therefore, we will need to rectify and buck the voltage down to a constant 5 Volts DC.
Seeing how important and potentially dangerous this process is, it is best to handle this with a designated IC.
Power Integrations makes a very easy to use IC called LinkSwitch-TN that will creat a non-isolated, low power 5 V rail.\\
I will simply use the reference design on their datasheet (for 12 V), but adjust the feedback resistors to achieve a 5V output instead.
This converter will only be able to produce about 1W of power, but that is all we will need to power the MCU and LCD.

\subsection{Display}
In order to display the set temperature and current temperature, we will need a set of 7-segment displays.
The 2 more common options is an LED display (often used in microwaves and alarm clocks) and a LCD (often used in themostats).
LED displays use more power, but can be seen in dark environments.
LCDs use much less power, but cannot be seen in dark environments.
Since dark viewing is not a requirement, the LCD is the more logical choice.\\

Since there will be a total of 16 segments that need to be chosen, we will either need 16 GPIOs or use a 4 to 16 decoder.
When using 16 GPIOs, the usage is obvious:  Simply set the voltage high when that segment is active.
When using a 4to16 decoder, the MCU will need to do some sequencing.
With the decoder, only 1 segment can be on at a time.  
THerefore, the MCU will need to quickly scan through each of the 16 segments and either activate or deactiveate.
This scanning will happen very quickly so the human eye will not be able to percieve the scan.

\subsection{Temperature Measurement}
There are 2 common ways of measuring temperature:  Thermistors and Thermocouples.
Thermistors are easier to measure (a simple voltage divider circuit can be used).
However, thermistor probes that are safe to put in water are a little pricey.
Thermocouples are dirt cheap (they are basically just a wire).  
However, they are harder to measure (an OpAmp circuit is required).
Since OpAmps are very cheap, using thermocouples will be the cheaper option.\\

A thermocouple is essentially a metal wire.
When any metal experience a gradient in temperature, it produces a low voltage (in the range of 40uV per degree Celcius).
Since we want to measure the temperature accurately to within 1 degree F, we will need an amplifier so the MCU's analog to digital converter can read the voltage.
The simpliest option would use a non-inverting opamp amplifer.\\

Lets do some math to find a good gain.  
The MCU is running at 5 Volts and has a 10bit ADC.  
THis means the smallest voltage it can detect is $ \frac{5}{2^{10}} = 4.88 mV $.
Assuming we use a thermocouple with a sensitivity of 40uV per degree, a change of 1 degree F will be
$1^oF * \frac{5^oC}{9^oF} * \frac{40 \mu V}{5^oC} = 22 \mu V$.
THerefore, in order to detect a shift in 1 degree F, a gain of $\frac{4.88 mV}{22 \mu V} = 221.8$.
Now, lets verify that this gain will not over-volt the ADC: $200^oF * 22\frac{\mu V}{F} * 221.8 = 0.975 V$.
So we have plenty of headroom to increase the gain.\\

Since thermocouples only measure temperature gradients, we will need to use an on-board thermistor to read the board temperature.
This temparature will be used as an offset to add to the thermocouple temperature.
Since it just measuring the baord temperature, it can be small and cheap.



\subsection{Coil Driver}
The heating coil is basically a 48 Ohm resistor rated to 300 Watts.
Therefore, it does not need an alternating voltage to run.
However, if we powered it with the rectified 170V DC rail, it would go way over its rated 300 Watts.
For this reason, its power will need to be switched on and off with a 50\% duty cycle to keep the power down.\\
\subsubsection{Switching CIrcuit}
Since we dont want this device to float at 170V when its off, we will use a PMOS to switch the power.
This will require a high-side FET driver.
I found that the cheapest method would be to create a custom floating fet driver circuit that uses a PNP and NPN to power the Gate.  A 10V Zener will be used to power the driver.
SInce the switching frequency will be low (around 120Hz), the zener can be low power.
This same circuit can also be used to turn on and off the pump if we dont want to leave it on 24/7.

\subsection{User Interface Buttons}
The circuit will require 3 buttons:  Increase Temp, Decrease Temp, Reset.
These switches can simply use the internal pull ups so no external circuitry will be needed.
To prevent switch bouncing, we will likely need to implement a debouncing routine in the software.
This will prevent a single button press from becoming a multi-click press.

\subsection{Microcontroller}
I will likely use an Atmel AtMega328.
It is not the cheapest option, but i am familiar with the platform because it is used by Arduino.
It has plenty of GPIOs and ADC ports.

\section{Software Design}
\subsection{PID Controller}
To regulate the water temperature, i will use a PID control loop.
PID stands for Proportional, Integral, Derivative.  
Like every control loop, PID monitors the error between the set temp and the current temp.
In addition, PID monitors the integral and derivitive of the error to quickly bring the error to zero without overshooting.\\

The hardest part will be picking the P, I and D constants. 
These constant are used to weight the importance of each input.  
Instead of doing the thermal dynamics to figure out good constants, i will probably just use trial and error to pick a conservative.\\

To start guessing a good set of PID constants, we should know how long it will take to heat the water.
Assuming we use my 5 Liter Crock Pot to hold th water, the Heating coil produces 300 Watts, and we are heating the water from 70 degrees F to 160 Degrees F (90 Degrees delta), it will take 58 minutes to heat up the water.
\begin{equation}
5 L * \frac{4.18 joules}{.001 L * 1^oC} * 90^oF * \frac{5^oC}{9^oF} * \frac{1 second}{300 joules} = 58 minutes
\end{equation}
This is a very weak estimation since it assumes the water is perfectly insulated from the surrounding air, but it gives us an idea of how long the heating cycle will take.

In a quick, dirty test, i found that the water temp in my crock pot drops about 2 degrees in 10 minutes.


\subsection{LCD Driver}
\subsection{Switch Debouncing}



\end{document}
